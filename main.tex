\documentclass[notheorems, hyperref={backref}]{beamer}

%% KEY LINES IN THIS TEX FILE %% (enter line number+gg to go)

%
% LOCAL FONT DEFINITIONS -- need to come first
%
%\usepackage{mathpazo}
\usepackage{libertine}
\usepackage[libertine]{newtxmath}
\usefonttheme[onlymath]{serif}

%
% STANDARD PREAMBLE
%
\input{preamble}
\allowdisplaybreaks

% Extra packages
\usepackage{relsize}

% TIkzlibraries
\usetikzlibrary{decorations.pathmorphing}

%
% ABOUT FONT DEFINTIONS IN THE PREAMBLE
%
% Mathscr for sheaves use \sA, where A can be any letter. Exceptions and additions:
% % \E (vector bundles)
% % \F (coherent sheaves)
% % \G (coherent sheaves)
% % \hom (sheaf hom)
% % \I (ideal sheaves)
% % \L (line bundles)
% % \M (line bundles)
% % \O (structure sheaf)
% % \w (canonical sheaf)
%
% Mathcal use \calA. Exceptions and additions:
% % \U (open cover)
% % \X (families of varieties)
% % \Y (families of varieties)
%
% Mathbb use \bbA. Exceptions and additions:
% % \A (affine space)
% % \C (complex numbers)
% % \Gm (puctured affine line)
% % \k (field)
% % \N (natural numbers)
% % \P (projective space)
% % \Q (rational numbers)
% % \R (real numbers)
% % \V (geometric vector bundle)
% % \Z (integers)
%
% Boldfont for categories use \bfA. Additions:
% % \Cat (categories)
% % \Coh (coherent sheaves)
% % \D (derived category)
% % \Db (bounded derived category)
% % \K (homotopy category)
% % \Mod (modules)
% % \PSh (presheaves)
% % \QCoh (quasi-coherent sheaves)
% % \Set (sets)
% % \Sh (sheaves)
% % \Top (topological spaces)
% % \Vec (vector bundles)
%
% Mathfrak for ideals
% % From \a to \e
% % \m and \n for maximal ideals

%
% THEOREM ENVIRONMENTS
%
\setbeamertemplate{theorems}[numbered]
\newtheorem*{thm}{Theorem}
\newtheorem{exe}{Exercise}

%
% THEOREM CROSS-REFERENCING
%
\crefname{thm}{theorem}{theorems}
\Crefname{thm}{Theorem}{Theorems}
\crefname{lm}{lemma}{lemmas}
\Crefname{lm}{Lemma}{Lemmas}
\crefname{prop}{proposition}{propositions}
\Crefname{prop}{Proposition}{Propositions}
\crefname{cor}{corollary}{corollaries}
\Crefname{cor}{Corollary}{Corollaries}
\crefname{conj}{conjecture}{conjectures}
\Crefname{conj}{Conjecture}{Conjectures}
\crefname{defn}{definition}{definitions}
\Crefname{defn}{Definition}{Definitions}
\crefname{exa}{example}{examples}
\Crefname{exa}{Example}{Examples}
\crefname{rem}{remark}{remarks}
\Crefname{rem}{Remark}{Remarks}
\crefname{nota}{notation}{notations}
\Crefname{nota}{Notation}{Notations}
\crefname{fact}{fact}{facts}
\Crefname{fact}{Fact}{Facts}
\crefname{q}{question}{questions}
\Crefname{q}{Question}{Questions}
\crefname{pbl}{problem}{problems}
\Crefname{pbl}{Problem}{Problems}

%
% MATH OPERATORS
%
\DeclareMathOperator{\Hom}{Hom}
\DeclareMathOperator{\Fib}{Fib}
\DeclareMathOperator{\id}{id}
\DeclareMathOperator{\Aut}{Aut}

%
% OTHER COMMANDS
%
\newcommand{\ot}{\otimes}
\newcommand{\op}{\oplus}
\newcommand{\Cov}{\mathbf{Cov}}
\newcommand{\act}{\mbox{ }\rotatebox[origin=c]{-120}{$\circlearrowright$}}

%
% TITLE PAGE INFORMATION
%
\title[Basics of monodromy]{Basics of Monodromy}
\author{Pedro Núñez}
\institute{Basic Notions --- University of Freiburg}
\date{14th May 2020}
 
%
% LINKS AND PDF OPTIONS
%
\makeatletter
\hypersetup{
  %pdfauthor={\authors},
  pdftitle={\@title},
  %pdfsubject={\@subjclass},
  %pdfkeywords={\@keywords},
  %pdfstartview={Fit},
  %pdfpagelayout={TwoColumnRight},
  %pdfpagemode={UseOutlines},
  bookmarks,
  colorlinks,
  linkcolor=linkblue,
  citecolor=linkred,
  urlcolor=linkred}
\makeatother

% Table of contents
\setbeamertemplate{section in toc}[sections numbered]

% Add table of contents at the beginning of each section
\AtBeginSection[]
{
    \begin{frame}
	\frametitle{Table of Contents}
	\tableofcontents[currentsection]
    \end{frame}
}
\usecolortheme{orchid}
 
\begin{document}
 
\frame{\titlepage}

\section{Introduction}
\begin{frame}
    \frametitle{Motivating example --- A multi-valued function}
    Consider $z=re^{\theta i}\mapsto z^{2}=r^{2}e^{2\theta i}$ on $\C$.
    Local inverse on $\C^{\times}$:
    \[ z=re^{\theta i} \mapsto \sqrt{z}=\sqrt{r}e^{\frac{\theta}{2}i}. \]
    \pause

    \textbf{Ambiguity:} the previous expression is not well defined, as
    \[ re^{\theta i}=re^{(\theta+2\pi)i}\mapsto \sqrt{r}e^{\frac{\theta}{2}i}\neq \sqrt{r}e^{\left(\frac{\theta}{2}+\pi\right)i}. \]
    \pause

    Let $z_{0}\in \C^{\times}$ and pick one value for $\sqrt{z_{0}}$.
    Let $\gamma\colon [0,1]\to \C^{\times}$ be a path with $\gamma(0)=z_{0}$.
    Then the chosen $\sqrt{z_{0}}$ determines uniquely a value of $\sqrt{\gamma(t)}$ for all $t\in [0,1]$, because we want $z\mapsto \sqrt{z}$ to be continuous.
\end{frame}
 
\begin{frame}
    \frametitle{The Monodromy Theorem}
    \begin{thm}[Weierstraß]
	Analytic continuation along a path only depends on the path up to homotopy.
    \end{thm}
    \pause
    \vspace{2em}
    In particular, if we walk around a simply connected space, then the analytic continuation is single-valued everywhere.
    \pause
    \vspace{2em}
    Hence:
    
    \textbf{``monodromy''}, \textit{mónos} (alone, only, single) and \textit{drómos} (running).
\end{frame}

\begin{frame}
    \frametitle{Polydromy, a.k.a.~lack of monodromy}
    Let's go back to our example $z\mapsto \sqrt{z}$ on $\C^{\times}$.
    \pause

    Pick a value of $\sqrt{1}$, say $1$ itself, and let $\gamma\colon [0,1]\to \C^{\times}$ be a loop at $1$ around $0$.
    Extend $\sqrt{z}$ along $\gamma$ as before.
    \pause
    \begin{figure}[htp]
	\centering
	\includegraphics[scale=.3]{pictures/root.jpg}
	\caption{After the loop we arrive at $-1$, the other possible value of $\sqrt{1}$.}
    \end{figure}
\end{frame}

\begin{frame}
    \frametitle{Why are we then talking about monodromy?}
    The Monodromy Theorem became so famous that people kept using the word ``monodromy'' to talk about polydromy\footnote{Frans Oort gave this explanation to Fabrizio Catanese.}.
    \pause
    \begin{figure}[htp]
	\centering
	\includegraphics[scale=.3]{pictures/RedHerring.png}
	\caption{This is an example of \href{https://ncatlab.org/nlab/show/red+herring+principle}{mathematical red herring principle}.}
    \end{figure}
    \pause
    \setcounter{exe}{-1}
    \begin{exe}
	This is the second red herring that appeared in this talk so far.

	Can you spot the first one?
    \end{exe}
\end{frame}

\begin{frame}
    \frametitle{Goal: understand polydromy}
    The Monodromy Theorem implies that $\pi_{1}(\C^{\times},z_{0})$ acts on the different possible values of $\sqrt{z_{0}}$.
    \pause

    The goal of this talk is to generalize this situation as follows:
    \pause
    \begin{enumerate}[label=\textbullet]
	\setlength\itemsep{1em}
	\item As we move $z$ in $\C^{\times}$, the possible values of $\sqrt{z}$ form a nice \textbf{covering space} of $\C^{\times}$.
	\pause
	\item If $p\colon Y\to X$ is a nice covering space, then $\pi_{1}(X,x)$ acts naturally on $p^{-1}(x)$.
	    This is the \textbf{monodromy action}.
	\pause
    \item We can recover the covering space from the monodromy action!
	\pause
    \item If the fibres of $p$ carry a natural vector space structure, then one can use the tools of \textbf{representation theory} to study polydromy.
	\pause
	This happens both naturally (e.g.~when solving differential equations on a complex domain) and artificially (e.g.~replacing the fibres by their cohomology groups).
    \end{enumerate}
\end{frame}

\section{Galois correspondence}
\begin{frame}
    \frametitle{Covering spaces}
    Let $X$ be a topological space.
    \pause
    \begin{enumerate}[label=\textbullet]
	\setlength\itemsep{1em}
	\item The category of \textit{spaces over $X$} has for objects (continuous) maps $p\colon Y\to X$ and for morphisms commutative triangles
	    \begin{center}
		\begin{tikzcd}[ampersand replacement=\&]
		    Y_{1}\arrow[swap]{dr}{p_{1}}\arrow{rr}{f} \& \& Y_{2}\arrow{dl}{p_{2}} \\
		    \& X \& 
		\end{tikzcd}
	    \end{center}
	    \pause
	\item A map $p\colon Y\to X$ has property $\mathbf{P}$ \textit{locally on $X$} if every point $x\in X$ has an open neighbourhood $x\in U\subseteq X$ such that $\mathbf{P}$ is true for $p|_{p^{-1}(U)}\colon p^{-1}(U)\to U$.
	    \pause
	\item $p\colon Y\to X$ is a \textit{covering space} if locally on $X$ it is isomorphic to a projection $X\times F\to X$ for some discrete space $F$.
    \end{enumerate}
\end{frame}

%\begin{frame}
%    \frametitle{Examples}
%    \begin{enumerate}[label=\textbullet]
%	\setlength\itemsep{1em}
%	\item Let $F$ be a discrete space.
%	    Then the projection $X\times F\to X$ is a cover, called a \textit{trivial cover} of $X$.
%	    \pause
%	\item Let $\Z$ act on $\R$ by translation $n\cdot x:=x+n$.
%	    Then the quotient map $\R\to \R/\Z=S^{1}$ is a cover.
%	    Could it be a trivial cover?
%	\item Consider again the map $z\mapsto z^{2}$.
%	    This does not define a cover over $\C$, because $0$ is a \textit{branching point}.
%	    But it does define a cover over $\C^{\times}$.
%    \end{enumerate}
%    \pause
%    \begin{exe}
%	A (continuous) group action $G\act Y$ is called \textit{even} if each $y\in Y$ has an open neighbourhood $y\in V\subseteq Y$ such that the sets $gV$ are pairwise disjoint for all $g\in G$.
%
%	If $G\act Y$ is even, then $p_{G}\colon Y\to G\backslash Y$ is a covering.
%    \end{exe}
%\end{frame}

\begin{frame}
    \frametitle{Maps into covering spaces} 
    \begin{figure}[htp] 
	\centering
	\includegraphics[scale=.25]{pictures/lemma.jpg}
	\caption{The set $\{z\in Z\mid f(z)=g(z)\}$ is open and closed, so if $Z$ is connected and $f$ and $g$ agree on a single point, then they agree in all of $Z$.}
    \end{figure}
    \pause
    In particular, if $p\colon Y\to X$ is a connected cover and $\phi\in \Aut(Y\mid X)$ fixes a point, then $\phi=\id_{Y}$.
\end{frame}

%\begin{frame}
%    \frametitle{Some more consequences}
%    \begin{enumerate}[label=\textbullet]
%	\setlength\itemsep{1em}
%	\item If $p\colon Y\to X$ is a connected cover, then $\Aut(Y\mid X)\act Y$ is even.
%	    \pause
%	\item We saw earlier that if $G\act Y$ is even, then $Y\to G\backslash Y$ is a cover.
%	    Suppose $Y$ is connected.
%	    \pause
%	    Define a group homomorphism:
%	    \begin{align*}
%		G & \longrightarrow \Aut(Y\mid G\backslash Y) \\
%		g & \longmapsto (y\mapsto g\cdot y)
%	    \end{align*}
%	    \pause
%	    \begin{enumerate}[label=$\circ$]
%		\item Since $G\act Y$ is even, $G\to \Aut(Y\mid G\backslash Y)$ is injective.
%		\pause
%
%		\item For $\phi\in \Aut(Y\mid G\backslash Y)$ and $y_{0}\in Y$, there is $g\in G$ with $\phi(y_{0})=g\cdot y_{0}$, because the fires are orbits and $\phi$ preservves fibres.
%		\pause
%		Since $Y$ is connected, $\phi(y)=g\cdot y$ for all $y\in Y$.
%	    \end{enumerate}
%    \end{enumerate}
%\end{frame}

\begin{frame}
    \frametitle{Galois correspondence [$X$ locally connected]}
    A connected cover $p\colon Y\to X$ is called \textit{Galois} if $X=\Aut(Y\mid X)\backslash Y$;
    \pause

    equivalently, if $\Aut(Y\mid X)$ acts transitively on each fibre.
    \pause

    \begin{thm}[{\cite[Theorem 2.2.10]{sza08}}]
	Let $p\colon Y\to X$ be a Galois cover.
	Then there is a bijection
	\begin{align*}
	    \left\{ \begin{matrix*}
		\mathrm{ Subgroups } \\
	    0\subseteq H\subseteq \Aut(Y\mid X)
	    \end{matrix*} \right\} & \leftrightarrow \left\{ \begin{matrix*}
		\mathrm{Connected} \\
		\mathrm{intermediate} \\
		\mathrm{covers}
	    \end{matrix*} 
	    \quad
	    \begin{tikzcd}[ampersand replacement=\&]
		Z\arrow[swap]{d}{q} \& Y\arrow{dl}{p}\arrow[dashed, swap]{l}{\exists} \\
		X \&
	    \end{tikzcd}
	    \right\} \\
	    H \quad  & \mapsto	\quad \left( H\backslash Y\to X\right) \\
	    \Aut(Y\mid Z) \quad & \mapsfrom \quad \left( Z\to X\right)
	\end{align*}
	\pause
	Moreover, $q\colon Z\to X$ is Galois if and only if $H\subseteq \Aut(Y\mid X)$ is a normal subgroup, in which case we have
	\[ \Aut(Z\mid X)\cong G/H . \]
    \end{thm}
\end{frame}

\begin{frame}
    \frametitle{Proof of the bijection in the previous theorem}
    \begin{enumerate}[label=\arabic*)]
	\item $H\backslash Y\to X$ is a cover (local on $X$, hence may assume $Y=X\times F$).
	    \pause
	\begin{exe}
	    A continuous action $G\act Y$ is called \normalfont{even} if each $y\in Y$ has an open nhood $y\in V\subseteq Y$ such that $\{gV\}_{g\in G}$ are pairwise disjoint.
	    Show that $Y\to G\backslash Y$ is then a cover and deduce that $Y\to H\backslash Y$ is a cover.
	\end{exe}
	    \pause
    \item Define $\varphi\colon H\to \Aut(Y\mid H\backslash Y)$ by $\varphi(h)(y):=h\cdot y$.
	Since $H\act Y$ is even, $\varphi$ is injective.
	By ``Maps into covering spaces'' it is also surjective.
	    Hence
	    \framebox{$H\mapsto (H\backslash Y\to X)\mapsto \Aut(Y\mid H\backslash Y)\cong H$}.
	    \pause
	\item If $q\colon Z\to X$ is an intermediate connected cover, then the map $f\colon Y\to Z$ is a cover as well (local on $Z$, hence on $X$, hence we may assume that this map has the form $X\times F_{Y}\to X\times F_{Z}$).
	    \pause
	\item Since $\Aut(Y\mid X)\act p^{-1}(q(z))$ is transitive, $\Aut(Y\mid Z)\act f^{-1}(z)$ is transitive as well by ``Maps into covering spaces''.
	    Hence $Y\to Z$ is Galois and \framebox{$Z\mapsto \Aut(Y\mid Z)\mapsto \Aut(Y\mid Z)\backslash Y=Z$}.
    \end{enumerate}
\end{frame}

\section{Monodromy action}
\begin{frame}
    \frametitle{Homotopy Lifting}
    Let $p\colon Y\to X$ be a cover and $f_{t}\colon Z\to X$ a
    
    homotopy, i.e.~a map $F\colon Z\times [0,1]\to X$.
    \pause

    If $\tilde{f}_{0}\colon Z\to Y$ is a lift of $f_{0}\colon Z\to X$, then 
    
    we can extend it to a lift $\tilde{F}\colon Z\times [0,1]\to Y$ of the whole homotopy.
    \pause
    \begin{tikzpicture}[remember picture,overlay]
    \node[xshift=-25mm,yshift=-16mm] at (current page.north east)
    {\includegraphics[scale=.22]{pictures/HippityHoppity.png}};
    \end{tikzpicture}
    \pause

    \begin{enumerate}[label=\arabic*)]
	\item Let $z_{0}\in W\subseteq Z$ be an open neighbourhood of a point in $Z$

	    for which there exists a subdivision $0=t_{0}<t_{1}<\cdots <t_{m}=1$

	    such that $p\colon Y\to X$ is trivial over $F(W\times [t_{i},t_{i+1}])$ for all $i$.
	    \pause
	\item Since $p$ is trivial over $F(W\times [0,t_{1}])$, there is a unique way to extend the lifting $\tilde{f}_{0}$ to liftings $\tilde{f}_{t}$ for $t\in [0,t_{1}]$.
	    \pause
	\item Iterate this process to obtain a local lifting $\tilde{F}\colon W\times [0,1]\to Y$.
	    \pause
	\item Do the same for each point $z\in Z$.
	    On the overlaps the extensions agree by ``Maps into covering spaces'' applied to each $\{z\}\times [0,1]$, because $\tilde{F}(z,0)$ has to be $\tilde{f}_{0}(z)$.
    \end{enumerate}
\end{frame}

\begin{frame}
    \frametitle{The monodromy action}
    \begin{figure}[htp]
	\centering
	\includegraphics[scale=.25]{pictures/monodromy.jpg}
	\caption{Given the class of a path $[\gamma]\in \pi_{1}(X,x)$ and a point $y\in p^{-1}(x)$, set $[\gamma]\cdot y:=\tilde{\gamma}(1)$, where $\tilde{\gamma}$ is the unique lift of $\gamma$ to the cover.
	Only defining concatenation the unconventional way we obtain a \textit{left} action!}
    \end{figure}
\end{frame}

\begin{frame}
    \frametitle{Cover$\leftrightarrow $Monodromy [$X$ connected+locally $1$-connected]}
    \begin{thm}[{\cite[Theorem 2.3.4]{sza08}}]
	The functor
	\begin{align*}
	    \Fib_{x}\colon \Cov(X) & \longrightarrow \pi_{1}(X,x)-\Set \\
	    (p\colon Y\to X) & \longmapsto \pi_{1}(X,x)\act p^{-1}(x)
	\end{align*}
	is an equivalence of categories from the category of covers of $X$ to the category of sets endowed with a $\pi_{1}(X,x)$-action.
    \end{thm}
    \pause

    \setcounter{exe}{1}
    \begin{exe}
	Check that $\Fib_{x}$ is a functor.
	[Hint: ``Maps into covering spaces''.]
    \end{exe}
\end{frame}

\begin{frame}
    \frametitle{Sketch of proof --- Part 1: the universal cover}
    \begin{enumerate}[label=\arabic*)]
	\item The \textit{universal cover} $\tilde{X}_{x}$ consists of homotopy classes of paths in $X$ starting at $X$, and the projection $\pi\colon \tilde{X}_{x}\to X$ is $\pi([\alpha]):=\alpha(1)$.
	    \pause
	\item Let $y\in p^{-1}(x)$.
	    Define $\pi_{y}\in \Hom_{X}(\tilde{X}_{x},Y)$ by $\pi_{y}([\alpha]):=\tilde{\alpha}(1)$.
	    \pause
	\item Let $\phi\in \Hom_{X}(\tilde{X}_{x},Y)$.
	    Define $y\in p^{-1}(x)$ as $y:=\phi([x])$.
	    \pause
	\item These two maps are mutually inverse, so \framebox{$\Fib_{x}\cong \Hom_{X}(\tilde{X}_{x},-)$}.
	    \pause
	\item $\pi\colon \tilde{X}_{x}\to X$ is Galois, because $\pi_{[\gamma]}$ is an automorphism and $\pi_{[\gamma]}([x])=[\gamma]$ (suffices to check transitivity on a single fibre).
	    \pause
	\item \framebox{$\pi_{1}(X,x)\cong \Aut(\tilde{X}_{x}\mid X)^{\mathrm{op}}$} via $[\gamma]\longmapsto ([\alpha]\mapsto [\alpha\cdot \gamma])$.
	    \pause
	\item Let $\phi\in \Aut(\tilde{X}_{x}\mid X)$ and $y\in p^{-1}(x)$.
	    Define $\phi\cdot y:=\pi_{y}\circ \phi([x])$, i.e.~the point in $\Fib_{x}(Y)$ corresponding to $\pi_{y}\circ \phi\in \Hom_{X}(\tilde{X}_{x},Y)$.
	    Then $\psi\cdot(\phi\cdot y)$ corresponds to $\pi_{y}\circ \phi\circ \psi=\pi_{y}\circ (\psi\circ^{\mathrm{op}}\phi)$.
	    We get $\Aut(\tilde{X}_{x}\mid X)^{\mathrm{op}}\act p^{-1}(x)$, which agrees with $\pi_{1}(X,x)\act p^{-1}(x)$.
    \end{enumerate}
\end{frame}

\begin{frame}
    \frametitle{Sketch of proof --- Part 2: fully faithfulness}
    \begin{enumerate}[label=\arabic*)]
	\item Let $\psi\colon \Fib_{x}(Y)\to \Fib_{x}(Z)$ be $G:=\pi_{1}(X,x)$-equivariant.
	    Want some $f\colon Y\to Z$ over $X$ such that $\psi(y)=f(y)$ for all $y\in p^{-1}(x)$.
	    \pause
	\item We may assume $Y$ and $Z$ connected ($G$-orbits are connected).
	    \pause
	\item Hence faithfulness follows from ``Maps into covering spaces''..
	    \pause
	\item Let $\pi_{y}\colon \tilde{X}_{x}\to Y$ corresponding to $y\in p^{-1}(x)$, so that \framebox{$Y=\Aut(\tilde{X}_{x}\mid Y)\backslash \tilde{X}_{x}$} by Galois correspondence.
	    \pause
	\item Via $G\cong \Aut(\tilde{X}_{x}\mid X)$ we can identify $G_{y}:=\{ g\in G\mid g\cdot y=y\}$ and $\{ \phi\in \Aut(\tilde{X}_{x}\mid X)\mid \pi_{y}\circ \phi =\pi_{y}\}$.
	    Hence \framebox{$\Aut(\tilde{X}_{x}\mid Y)=G_{y}$}.
	    \pause
	\item $G_{y}\subseteq G_{\psi(y)}$ by $G$-equivar.~$\rightsquigarrow f\colon Y=G_{y}\backslash \tilde{X}_{x}\to G_{\psi(y)}\backslash \tilde{X}_{x}=Z$.
	\item $f(y)=f(\pi_{y}([x]))=\pi_{\psi(y)}([x])=\psi(y)$.
	    For $y'\in p^{-1}(y)$, let $\gamma$ s.t.~$[\gamma]\cdot y=y'$, so that $\psi(y')=[\gamma]\cdot \psi(y)$.
	    Then we have
	    \[ f(y')=f\circ\tilde{\gamma}^{Y}(1)=\tilde{\gamma}^{Z}(1)=[\gamma]\cdot \psi(y)=\psi(y'). \]
    \end{enumerate}
\end{frame}

\begin{frame}
    \frametitle{Sketch of proof --- Part 3: essential surjectivity}
    \begin{enumerate}[label=\arabic*)]
	\item Let $S$ be a set with a $G:=\pi_{1}(X,x)$ action.
	    \pause
	\item We may assume $G\act S$ transitive (otherwise split into orbits).
	    \pause
	\item Let $s\in S$ any and let $G_{s}$ be its stabiliser.
	    By Galois correspondence we can find
	    \begin{center}
		\begin{tikzcd}[ampersand replacement=\&]
		    \tilde{X}_{x}\arrow[swap]{dr}{\pi}\arrow{r}{q} \& Y:=G_{s}\backslash \tilde{X}_{x}\arrow{d}{p} \\
		    \& X
		\end{tikzcd}
	    \end{center}
	    \pause
	\item Since $p\colon Y\to X$ is connected, $G\act p^{-1}(x)$ is transitive.
	\item Define $\varphi(q([x])):=s$.
	    We will try to extend by $G$-equivariance.
	    \pause
	\item If $g'\cdot q([x])=g\cdot q([x])$, then $g'g^{-1}\in \Aut(\tilde{X}_{x}\mid Y)=G_{s}$.
	    Hence $\varphi\colon p^{-1}(x) \to S$ defined as $y=g\cdot q([x]) \mapsto g\cdot s$ is $G$-equivar.
	    \pause
	\item If $g\cdot s=g'\cdot s$, then again $g'g^{-1}\in G_{s}=G_{q([x])}$, so $g\cdot q([x])=g'\cdot q([x])$.
	    And transitivity implies surjectivity.
    \end{enumerate}
\end{frame}

\section{Examples}
\begin{frame}
    \frametitle{Back to our motivating example}
\end{frame}

\begin{frame}
    \frametitle{Cover of Riemann surface}
\end{frame}

\begin{frame}
    \frametitle{Some complex analysis}
\end{frame}

\end{document}
