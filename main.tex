\documentclass[notheorems, hyperref={backref}]{beamer}

%% KEY LINES IN THIS TEX FILE %% (enter line number+gg to go)

%
% LOCAL FONT DEFINITIONS -- need to come first
%
%\usepackage{mathpazo}
\usepackage{libertine}
\usepackage[libertine]{newtxmath}
\usefonttheme[onlymath]{serif}

%
% STANDARD PREAMBLE
%
\input{preamble}
\allowdisplaybreaks

% Extra packages
\usepackage{relsize}

% TIkzlibraries
\usetikzlibrary{decorations.pathmorphing}

%
% ABOUT FONT DEFINTIONS IN THE PREAMBLE
%
% Mathscr for sheaves use \sA, where A can be any letter. Exceptions and additions:
% % \E (vector bundles)
% % \F (coherent sheaves)
% % \G (coherent sheaves)
% % \hom (sheaf hom)
% % \I (ideal sheaves)
% % \L (line bundles)
% % \M (line bundles)
% % \O (structure sheaf)
% % \w (canonical sheaf)
%
% Mathcal use \calA. Exceptions and additions:
% % \U (open cover)
% % \X (families of varieties)
% % \Y (families of varieties)
%
% Mathbb use \bbA. Exceptions and additions:
% % \A (affine space)
% % \C (complex numbers)
% % \Gm (puctured affine line)
% % \k (field)
% % \N (natural numbers)
% % \P (projective space)
% % \Q (rational numbers)
% % \R (real numbers)
% % \V (geometric vector bundle)
% % \Z (integers)
%
% Boldfont for categories use \bfA. Additions:
% % \Cat (categories)
% % \Coh (coherent sheaves)
% % \D (derived category)
% % \Db (bounded derived category)
% % \K (homotopy category)
% % \Mod (modules)
% % \PSh (presheaves)
% % \QCoh (quasi-coherent sheaves)
% % \Set (sets)
% % \Sh (sheaves)
% % \Top (topological spaces)
% % \Vec (vector bundles)
%
% Mathfrak for ideals
% % From \a to \e
% % \m and \n for maximal ideals

%
% THEOREM ENVIRONMENTS
%
\setbeamertemplate{theorems}[numbered]
\newtheorem{thm}{Theorem}
\newtheorem{exe}{Exercise}

%
% THEOREM CROSS-REFERENCING
%
\crefname{thm}{theorem}{theorems}
\Crefname{thm}{Theorem}{Theorems}
\crefname{lm}{lemma}{lemmas}
\Crefname{lm}{Lemma}{Lemmas}
\crefname{prop}{proposition}{propositions}
\Crefname{prop}{Proposition}{Propositions}
\crefname{cor}{corollary}{corollaries}
\Crefname{cor}{Corollary}{Corollaries}
\crefname{conj}{conjecture}{conjectures}
\Crefname{conj}{Conjecture}{Conjectures}
\crefname{defn}{definition}{definitions}
\Crefname{defn}{Definition}{Definitions}
\crefname{exa}{example}{examples}
\Crefname{exa}{Example}{Examples}
\crefname{rem}{remark}{remarks}
\Crefname{rem}{Remark}{Remarks}
\crefname{nota}{notation}{notations}
\Crefname{nota}{Notation}{Notations}
\crefname{fact}{fact}{facts}
\Crefname{fact}{Fact}{Facts}
\crefname{q}{question}{questions}
\Crefname{q}{Question}{Questions}
\crefname{pbl}{problem}{problems}
\Crefname{pbl}{Problem}{Problems}

%
% MATH OPERATORS
%
\DeclareMathOperator{\Hom}{Hom}
\DeclareMathOperator{\id}{id}
\DeclareMathOperator{\Aut}{Aut}

%
% OTHER COMMANDS
%
\newcommand{\ot}{\otimes}
\newcommand{\op}{\oplus}
\newcommand{\act}{\mbox{ }\rotatebox[origin=c]{-120}{$\circlearrowright$}}

%
% TITLE PAGE INFORMATION
%
\title[Basics of monodromy]{Basics of Monodromy}
\author{Pedro Núñez}
\institute{Basic Notions --- University of Freiburg}
\date{14th May 2020}
 
%
% LINKS AND PDF OPTIONS
%
\makeatletter
\hypersetup{
  %pdfauthor={\authors},
  pdftitle={\@title},
  %pdfsubject={\@subjclass},
  %pdfkeywords={\@keywords},
  %pdfstartview={Fit},
  %pdfpagelayout={TwoColumnRight},
  %pdfpagemode={UseOutlines},
  bookmarks,
  colorlinks,
  linkcolor=linkblue,
  citecolor=linkred,
  urlcolor=linkred}
\makeatother

% Table of contents
\setbeamertemplate{section in toc}[sections numbered]

% Add table of contents at the beginning of each section
\AtBeginSection[]
{
    \begin{frame}
	\frametitle{Table of Contents}
	\tableofcontents[currentsection]
    \end{frame}
}
 
\begin{document}
 
\frame{\titlepage}

\section{Introduction}
\begin{frame}
    \frametitle{Motivating example --- A multi-valued function}
    Consider $z=re^{\theta i}\mapsto z^{2}=r^{2}e^{2\theta i}$ on $\C$.
    Local inverse on $\C^{\times}$:
    \[ z=re^{\theta i} \mapsto \sqrt{z}=\sqrt{r}e^{\frac{\theta}{2}i}. \]
    \pause

    \textbf{Ambiguity:} the previous expression is not well defined, as
    \[ re^{\theta i}=re^{(\theta+2\pi)i}\mapsto \sqrt{r}e^{\frac{\theta}{2}i}\neq \sqrt{r}e^{\left(\frac{\theta}{2}+\pi\right)i}. \]
    \pause

    Let $z_{0}\in \C^{\times}$ and pick one value for $\sqrt{z_{0}}$.
    Let $\gamma\colon [0,1]\to \C^{\times}$ be a path with $\gamma(0)=z_{0}$.
    Then the chosen $\sqrt{z_{0}}$ determines uniquely a value of $\sqrt{\gamma(t)}$ for all $t\in [0,1]$, because we want $z\mapsto \sqrt{z}$ to be continuous.
\end{frame}
 
\begin{frame}
    \frametitle{The Monodromy Theorem}
    \begin{thm}[Weierstraß]
	Analytic continuation along a path only depends on the path up to homotopy.
    \end{thm}
    \pause
    \vspace{2em}
    In particular, if we walk around a simply connected space, then the analytic continuation is single-valued everywhere.
    \pause
    \vspace{2em}
    Hence:
    
    \textbf{``monodromy''}, \textit{mónos} (alone, only, single) and \textit{drómos} (running).
\end{frame}

\begin{frame}
    \frametitle{Polydromy, a.k.a.~lack of monodromy}
    Let's go back to our example $z\mapsto \sqrt{z}$ on $\C^{\times}$.
    \pause

    Pick a value of $\sqrt{z}$ at $z_{0}$ as before and let $\gamma\colon [0,1]\to \C^{\times}$ be a loop at $z_{0}$, with $\gamma(0)=\gamma(1)=z_{0}$.
    Extend $\sqrt{z}$ along $\gamma$ as before.
    \vspace{10em}
    \pause

    Do we always arrive at the same value of $\sqrt{z_{0}}$ at the end of the loop?
\end{frame}

\begin{frame}
    \frametitle{Why are we then talking about monodromy?}
    The Monodromy Theorem became so famous that people kept using the word ``monodromy'' to talk about polydromy\footnote{Fras Oort gave this explanation to Fabrizio Catanese.}.
    \pause
    \begin{figure}[htp]
    
	\centering
	\includegraphics[scale=.3]{pictures/RedHerring.png}
	\caption{This is an example of \href{https://ncatlab.org/nlab/show/red+herring+principle}{mathematical red herring principle}.}
    \end{figure}
    \pause
    \setcounter{exe}{-1}
    \begin{exe}
	This is the second red herring that appeared in this talk so far.

	Can you spot the first one?
    \end{exe}
\end{frame}

\begin{frame}
    \frametitle{Goal: understand polydromy}
    The Monodromy Theorem implies that $\pi_{1}(\C^{\times},z_{0})$ acts on the different possible values of $\sqrt{z_{0}}$.
    \pause

    The goal of this talk is to generalize this situation as follows:
    \pause
    \begin{enumerate}[label=\textbullet]
	\setlength\itemsep{1em}
	\item As we move $z$ in $\C^{\times}$, the possible values of $\sqrt{z}$ form a nice \textbf{covering space} of $\C^{\times}$.
	\pause
	\item If $p\colon Y\to X$ is a nice covering space, then $\pi_{1}(X,x)$ acts naturally on $p^{-1}(x)$.
	    This is the \textbf{monodromy action}.
	\pause
    \item If the fibres of $p$ carry a natural vector space structure, we will be able to use the tools of \textbf{representation theory} to study polydromy.
    \end{enumerate}
\end{frame}

\section{Galois coverings}
\begin{frame}
    \frametitle{Covering spaces}
    Let $X$ be a topological space.
    \pause
    \begin{enumerate}[label=\textbullet]
	\setlength\itemsep{1em}
	\item The category of \textit{spaces over $X$} has for objects (continuous) maps $p\colon Y\to X$ and for morphisms commutative triangles
	    \begin{center}
		\begin{tikzcd}[ampersand replacement=\&]
		    Y_{1}\arrow[swap]{dr}{p_{1}}\arrow{rr}{f} \& \& Y_{2}\arrow{dl}{p_{2}} \\
		    \& X \& 
		\end{tikzcd}
	    \end{center}
	    \pause
	\item A map $p\colon Y\to X$ has property $\mathbf{P}$ \textit{locally on $X$} if every point $x\in X$ has an open neighbourhood $x\in U\subseteq X$ such that $\mathbf{P}$ is true for $p|_{p^{-1}(U)}\colon p^{-1}(U)\to U$.
	    \pause
	\item $p\colon Y\to X$ is a \textit{covering space} if locally on $X$ it is isomorphic to a projection $X\times F\to X$ for some discrete space $F$.
    \end{enumerate}
\end{frame}

%\begin{frame}
%    \frametitle{Examples}
%    \begin{enumerate}[label=\textbullet]
%	\setlength\itemsep{1em}
%	\item Let $F$ be a discrete space.
%	    Then the projection $X\times F\to X$ is a cover, called a \textit{trivial cover} of $X$.
%	    \pause
%	\item Let $\Z$ act on $\R$ by translation $n\cdot x:=x+n$.
%	    Then the quotient map $\R\to \R/\Z=S^{1}$ is a cover.
%	    Could it be a trivial cover?
%	\item Consider again the map $z\mapsto z^{2}$.
%	    This does not define a cover over $\C$, because $0$ is a \textit{branching point}.
%	    But it does define a cover over $\C^{\times}$.
%    \end{enumerate}
%    \pause
%    \begin{exe}
%	A (continuous) group action $G\act Y$ is called \textit{even} if each $y\in Y$ has an open neighbourhood $y\in V\subseteq Y$ such that the sets $gV$ are pairwise disjoint for all $g\in G$.
%
%	If $G\act Y$ is even, then $p_{G}\colon Y\to G\backslash Y$ is a covering.
%    \end{exe}
%\end{frame}

\begin{frame}
    \frametitle{Maps into covering spaces} 
    \begin{figure}[htp] 
	\centering
	\includegraphics[scale=.25]{pictures/lemma.jpg}
	\caption{The set $\{z\in Z\mid f(z)=g(z)\}$ is open and closed, so if $Z$ is connected and $f$ and $g$ agree on a single point, then they agree in all of $Z$.}
    \end{figure}
    \pause
    In particular, if $p\colon Y\to X$ is a connected cover and $\phi\in \Aut(Y\mid X)$ fixes a point, then $\phi=\id_{Y}$.
\end{frame}

%\begin{frame}
%    \frametitle{Some more consequences}
%    \begin{enumerate}[label=\textbullet]
%	\setlength\itemsep{1em}
%	\item If $p\colon Y\to X$ is a connected cover, then $\Aut(Y\mid X)\act Y$ is even.
%	    \pause
%	\item We saw earlier that if $G\act Y$ is even, then $Y\to G\backslash Y$ is a cover.
%	    Suppose $Y$ is connected.
%	    \pause
%	    Define a group homomorphism:
%	    \begin{align*}
%		G & \longrightarrow \Aut(Y\mid G\backslash Y) \\
%		g & \longmapsto (y\mapsto g\cdot y)
%	    \end{align*}
%	    \pause
%	    \begin{enumerate}[label=$\circ$]
%		\item Since $G\act Y$ is even, $G\to \Aut(Y\mid G\backslash Y)$ is injective.
%		\pause
%
%		\item For $\phi\in \Aut(Y\mid G\backslash Y)$ and $y_{0}\in Y$, there is $g\in G$ with $\phi(y_{0})=g\cdot y_{0}$, because the fires are orbits and $\phi$ preservves fibres.
%		\pause
%		Since $Y$ is connected, $\phi(y)=g\cdot y$ for all $y\in Y$.
%	    \end{enumerate}
%    \end{enumerate}
%\end{frame}

\begin{frame}
    \frametitle{Galois coverings}
    A connected cover $p\colon Y\to X$ is called \textit{Galois} if $X=\Aut(Y\mid X)\backslash Y$;
    \pause

    equivalently, if $\Aut(Y\mid X)$ acts transitively on each fibre.
    \pause

    \begin{thm}
	Let $p\colon Y\to X$ be a Galois cover.
	Then there is a bijection
	\begin{align*}
	    \left\{ \mbox{ }\mathrm{Subgroups}\mbox{ }H\subseteq \Aut(Y\mid X) \mbox{ } \right\} & \leftrightarrow \left\{ \mbox{ }\mathrm{Intermediate} \mbox{ }\mathrm{covers} \mbox{ }q\colon Z\to X \mbox{ }\right\} \\
	    H \quad  & \mapsto	\quad \left( H\backslash Y\to X\right) \\
	    \Aut(Y\mid Z) \quad & \mapsfrom \quad \left( Z\to X\right)
	\end{align*}
	Moreover, $q\colon Z\to X$ is Galois if and only if $H\subseteq \Aut(Y\mid X)$ is a normal subgroup, in which case we have
	\[ \Aut(Z\mid X)=G/H . \]
    \end{thm}
\end{frame}

\section{Monodromy action}
\begin{frame}
    \frametitle{Monodromy action}
\end{frame}

\section{Local systems}
\begin{frame}
    \frametitle{Local systems}
\end{frame}

\end{document}
